% !TeX program = xelatex
\documentclass[
	fontsize=10pt, % Base font size
	oneside=true,
	twoside=false, % Use different layouts for even and odd pages (in particular, if twoside=true, the margin column will be always on the outside)
	open=any, % If twoside=true, uncomment this to force new chapters to start on any page, not only on right (odd) pages
	chapterprefix=true, % Uncomment to use the word "Chapter" before chapter numbers everywhere they appear
	%chapterentrydots=true, % Uncomment to output dots from the chapter name to the page number in the table of contents
	numbers=noenddot, % Comment to output dots after chapter numbers; the most common values for this option are: enddot, noenddot and auto (see the KOMAScript documentation for an in-depth explanation)
	%draft=true, % If uncommented, rulers will be added in the header and footer
	%overfullrule=true, % If uncommented, overly long lines will be marked by a black box; useful for correcting spacing problems
]{kaobook}

% support for languages
\usepackage{polyglossia}
\setmainlanguage[spelling=modern]{russian}
\setotherlanguages{english,japanese}
%\setmainfont[Ligatures=TeX]{CMU Serif}
%\setsansfont{CMU Sans Serif}
%\setmonofont{CMU Typewriter Text}
%\newfontfamily\cyrillicfont{CMU Serif}

%\usepackage{showframe} % Uncomment to show boxes around the text area, margin, header and footer
%\usepackage{showlabels} % Uncomment to output the content of \label commands to the document where they are used

% Bibliography/images/index
\usepackage{styles/environments}
\usepackage{styles/mdftheorems}
\graphicspath{{images/}{./}} % Paths in which to look for images
\makeindex[columns=3, title=Alphabetical Index, intoc] % Make LaTeX produce the files required to compile the index
\makeglossaries % Make LaTeX produce the files required to compile the glossary
\makenomenclature % Make LaTeX produce the files required to compile the nomenclature

% other important packages
\usepackage{comment}
\usepackage{pgffor} % enumerate lecture files
%\usepackage{multirow}
%\usepackage{caption}
\let\bf\bfseries
\let\it\itshape
\let\sf\sffamily

% book info
\titlehead{Анализ. Третий семестр}
\subject{Математический анализ}
\title{Конспект лекций С.В.~Кислякова}
\subtitle{}
\author[]{}
\date{\today}
\publishers{МКН СПбГУ}


\begin{document}
\frontmatter

%	OPENING PAGE
% \makeatletter
% \extratitle{
% 	% In the title page, the title is vspaced by 9.5\baselineskip
% 	\vspace*{9\baselineskip}
% 	\vspace*{\parskip}
% 	\begin{center}
% 		% In the title page, \huge is set after the komafont for title
% 		\usekomafont{title}\huge\@title
% 	\end{center}
% }
% \makeatother

%	COPYRIGHT PAGE
\makeatletter
\uppertitleback{\@titlehead} % Header
\lowertitleback{
	%\textbf{No copyright} \\
	%\cczero\ This book is released into the public domain using the CC0 code. To the extent possible under law, I waive all copyright and related or neighbouring rights to this work.

	%To view a copy of the CC0 code, visit: %\\\url{http://creativecommons.org/publicdomain/zero/1.0/}

	%\doclicenseThis

	\medskip

	\textbf{Colophon} \\
	This document was typeset with the help of \href{https://sourceforge.net/projects/koma-script/}{\KOMAScript} and \href{https://www.latex-project.org/}{\LaTeX} using the \href{https://github.com/fmarotta/kaobook/}{kaobook} class.

	\medskip

	\textbf{Publisher} \\
	First printed in May 2019 by \@publishers
}
\makeatother

%	OUTPUT TITLE PAGE AND PREVIOUS
% Note that \maketitle outputs the pages before here
% If twoside=false, \uppertitleback and \lowertitleback are not printed
% To overcome this issue, we set twoside=semi just before printing the title pages, and set it back to false just after the title pages
\KOMAoptions{twoside=semi}
\maketitle
\KOMAoptions{twoside=false}

% TABLE OF CONTENTS & LIST OF FIGURES/TABLES
\begingroup % Local scope for the following commands
% Define the style for the TOC, LOF, and LOT
%\setstretch{1} % Uncomment to modify line spacing in the ToC
%\hypersetup{linkcolor=blue} % Uncomment to set the colour of links in the ToC
%\setlength{\textheight}{23cm} % Manually adjust the height of the ToC pages

% Turn on compatibility mode for the etoc package
\etocstandarddisplaystyle % "toc display" as if etoc was not loaded
\etocstandardlines % "toc lines as if etoc was not loaded

\tableofcontents % Output the table of contents
%\listoffigures % Output the list of figures
% Comment both of the following lines to have the LOF and the LOT on different pages
%\let\cleardoublepage\bigskip
%\let\clearpage\bigskip
%\listoftables % Output the list of tables
\endgroup

%	MAIN BODY
\mainmatter % Denotes the start of the main document content, resets page numbering and uses arabic numbers
\setchapterstyle{kao} % Choose the default chapter heading style

% chapters
\foreach \n in {1} {\input{lec\n.tex}}
%----------------------------------------------------------------------------

\backmatter % Denotes the end of the main document content
\setchapterstyle{plain} % Output plain chapters from this point onwards

\begin{comment}
%	BIBLIOGRAPHY
% The bibliography needs to be compiled with biber using your LaTeX editor, or on the command line with 'biber main' from the template directory
%\defbibnote{bibnote}{Here are the references in citation order.\par\bigskip} % Prepend this text to the bibliography
\printbibliography[
	heading=bibintoc,
	title=Список литературы,
	%prenote=bibnote
] % Add the bibliography heading to the ToC, set the title of the bibliography and output the bibliography note

%	NOMENCLATURE
% The nomenclature needs to be compiled on the command line with 'makeindex main.nlo -s nomencl.ist -o main.nls' from the template directory
\nomenclature{$c$}{Speed of light in a vacuum inertial frame}
\nomenclature{$h$}{Planck constant}
\renewcommand{\nomname}{Notation} % Rename the default 'Nomenclature'
\renewcommand{\nompreamble}{The next list describes several symbols that will be later used within the body of the document.} % Prepend this text to the nomenclature
\printnomenclature % Output the nomenclature
\end{comment}

\begin{comment}
%	GLOSSARY
% The glossary needs to be compiled on the command line with 'makeglossaries main' from the template directory
\newglossaryentry{computer}{
	name=computer,
	description={is a programmable machine that receives input, stores and manipulates data, and provides output in a useful format}
}

% Glossary entries (used in text with e.g. \acrfull{fpsLabel} or \acrshort{fpsLabel})
\newacronym[longplural={Frames per Second}]{fpsLabel}{FPS}{Frame per Second}
\newacronym[longplural={Tables of Contents}]{tocLabel}{TOC}{Table of Contents}

\setglossarystyle{listgroup} % Set the style of the glossary (see https://en.wikibooks.org/wiki/LaTeX/Glossary for a reference)
\printglossary[title=Special Terms, toctitle=List of Terms] % Output the glossary, 'title' is the chapter heading for the glossary, toctitle is the table of contents heading
\end{comment}

%	INDEX
% The index needs to be compiled on the command line with 'makeindex main' from the template directory

\printindex % Output the index

%	BACK COVER
% If you have a PDF/image file that you want to use as a back cover, uncomment the following lines

%\clearpage
%\thispagestyle{empty}
%\null%
%\clearpage
%\includepdf{cover-back.pdf}

\end{document}
